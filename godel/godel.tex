%!TEX output_directory = .aux
%!TEX copy_output_on_build(true)

\documentclass[a4paper,10pt]{article}
\usepackage[top=2.5cm,bottom=2.5cm,left=2.5cm,right=2.5cm]{geometry}
\usepackage{xcolor,fancyhdr}
\usepackage[utf8]{inputenc}
\usepackage{amsfonts}
\usepackage{amssymb}
\usepackage{amsmath}
\usepackage{mathtools}
\usepackage{amsthm}
\usepackage{tikz}
\usepackage{courier}
\usepackage{stmaryrd}
\include{stylefile}
\usepackage{lineno}
\usepackage{algorithm}
\usepackage{algpseudocode}
\usepackage{nath}
\delimgrowth=1
\linenumbers
\mathindent = -1pt
\setpagewiselinenumbers

\newbox\gnBoxA
\newdimen\gnCornerHgt
\setbox\gnBoxA=\hbox{$\ulcorner$}
\global\gnCornerHgt=\ht\gnBoxA
\newdimen\gnArgHgt
\def\gnum #1{%
\setbox\gnBoxA=\hbox{$#1$}%
\gnArgHgt=\ht\gnBoxA%
\ifnum     \gnArgHgt<\gnCornerHgt \gnArgHgt=0pt%
\else \advance \gnArgHgt by -\gnCornerHgt%
\fi \raise\gnArgHgt\hbox{$\ulcorner$} \box\gnBoxA %
\raise\gnArgHgt\hbox{$\urcorner$}}

\makeatletter
\providecommand*{\Dashv}{%
  \mathrel{%
    \mathpalette\@Dashv\vDash
  }%
}
\newcommand*{\@Dashv}[2]{%
  \reflectbox{$\m@th#1#2$}%
}
\makeatother

\title{Gödel's Incompleteness Theorems}
\author{Giannis Tyrovolas}
\date{\today}
\theoremstyle{definition}
\newtheorem{theorem}{Theorem}
\newtheorem{prop}[theorem]{Proposition}
\newtheorem*{remark}{Remark}
\DeclarePairedDelimiter\abs{\lvert}{\rvert}
\DeclarePairedDelimiter\norm{\lVert}{\rVert}

\newtheorem{definition}[theorem]{Definition}
\newtheorem{example}[theorem]{Example}
\newtheorem{corollary}[theorem]{Corollary}
\newtheorem{lemma}[theorem]{Lemma}
\newtheorem{proposition}[theorem]{Proposition}
\newtheorem*{idea}{Idea}

\let\vec\mathbf
\let\phi\varphi
\newcommand{\proves}{\vdash}
\newcommand{\zero}{\overline{0}}
\newcommand{\nat}{\mathbb{N}}
\newcommand{\PA}{\mathrm{PA}}
\newcommand{\ProvS}[2][S]{{\mathrm{Pr}_{#1}{(\overline{\gnum{#2}})}}}
\let\Pr\ProvS

\begin{document}

We work in the language ${L}_E = \{\overline{0}, \,{}^+, \,v,\, f,\, ',\, (,\, ),\, -,\, \rightarrow, \,\forall, \,= ,\, \leqslant, \,\#\}$

\begin{definition}
    A subset $A \subseteq \mathbb{N}^k$ is \emph{definable} if there is a formula $\varphi(v_1, \ldots, v_k)$ such that 
    \[(n_1, \ldots, n_k) \in A \iff \varphi(\overline{n_1}, \ldots, \overline{n_k})\]
\end{definition}

\begin{definition}
    A subset $A \subseteq \mathbb{N}^k$ is \emph{provably definable} if there is $\varphi(\vec{x})$ such that $S \vdash \varphi(\vec{n}) \iff \vec{n} \in A$ and $S \vdash \neg\varphi(\vec{n}) \iff \vec{n} \notin A$ 
\end{definition}

\begin{definition}
    A function $f \colon \mathbb{N}^k \longrightarrow \mathbb{N}$ is \emph{definable} if $A = \{\vec{x}, f(\vec{x}) \mid \vec{x} \in \mathbb{N}^k\}$ is definable.\\
    It is \emph{weakly provably definable} from $S$ if $A$ is provably definable from $S$.\\
    It is \emph{provably definable} if for all $\vec{n} \in \mathbb{N}^k$, $S \vdash \forall v \left(\varphi(\overline{\vec{n}}, v) \leftrightarrow  f(\overline{\vec{n}}) = v\right)$
\end{definition}

\begin{definition}
    \begin{enumerate}
        \item $^{+} \colon \mathbb{N} \longrightarrow \mathbb{N} \setminus \{0\}$ is injective.
        \item Adding and multiplying by 0 on the right: $\forall v \left(v + \zero =\zero \right)$ and $\forall v \left(v \times \zero = \zero\right)$
        \item Addition, multiplication: $ \forall v_1 \forall v_2 (v_1 + v_2^+ = (v_1 + v_2)^+)$ and $\forall v_1 \forall v_2 (v_1 \times v_2^+ = v_1 \times v_2 + v_2)$.
        \item Relation $\leqslant$ is a total order, $\zero$ is the least element, $n^+$ is the successor of $n$.
        \item For any formula $\varphi(x)$ in one variable:
        \[
            (\varphi(\zero) \land \forall v_0(\varphi(v_0) \rightarrow \phi(v_0^+))) \rightarrow \forall v_0 (\phi(v_0))
        \]
    \end{enumerate}
\end{definition}

\begin{definition}
    For $\phi = \sigma_0 \ldots \sigma_n$ a formula of $L$, $  \gnum \phi = \displayed {\sum_{i=0}^n} \gnum {\sigma_i} 13^i$
\end{definition}

\subsection*{Building $\Pr{\phi}$}

\begin{enumerate}
    \item Syntax: $\lfloor\root{13}{n} \rfloor$, \begin{tight} ${k {+}{+} l}$ \end{tight}, $k$ is prefix/suffix/substring of $n$ and \emph{formula sequence} last of which is $\sigma$.
    \item Define \texttt{isNumeral} and \texttt{isVariable} by $\exists$. Define \texttt{isTerm} by valid sequence of term construction.
    \item Identify formulas: \texttt{isAtomic}, and \texttt{isAxiomFirstOrder}.
    \item So for $S$ a definable set of formulas in $\Delta_i$, $\mathrm{proof}_S(\overline{n}, \overline{m})$, is $\Delta_i$. $\Pr{\phi} = (\exists x) \mathrm{proof}_S(\overline{\gnum{\phi}}, x)$.
    \item Define $\PA$ in $\Delta_1$, we need the exists for the induction scheme. 
\end{enumerate}

\begin{definition}[Quasi-substitution]
    For $\phi(v_i)$ and term $t$ let $\phi[t] = \forall v_i (v_i = t \rightarrow \phi)$. We have $\PA \proves \phi(t) \leftrightarrow \phi[t]$.
\end{definition}

\begin{definition}
    $\Sigma_0 = \Pi_0 = \Delta_0$ formulas without unbounded quantifiers.
    $\Sigma_{n+1}$: formulas of the form $\exists x \phi (x)$, with $\phi \in \Pi_n$.
    Similarly, $\Pi_{n+1}$ is the formulas of the form $\forall x \phi(x)$ with $\phi \in \Sigma_n$. \\
    A formula $\psi$ is provably $\Sigma_n$ from $S$ if there is a $\phi \in \Sigma_n$, such that $S \vdash \psi \leftrightarrow \phi$.
\end{definition}

\begin{lemma}[Diagonal Lemma]
    For any formula $F(v_1)$ there is a formula $C$ such that:
    \[ \centering
        \PA \proves F(\overline{\gnum C}) \leftrightarrow C
    \]
\end{lemma}
Let $E_n$ the expression with Gödel number $n$. \\
    Let $d(n)$ be $E_n[\overline{n}]$ and $D(m, n)$ be the formula $n = \gnum{d(m)}$. \\
    Consider, $F(\overline{\gnum{y}})$, then $F(\overline{\gnum{d(y)}}) \vdash \dashv \psi(y) = \forall z(D(y,z) \rightarrow F(z))$. Let $k = \gnum{\psi}$, $C = \psi[\overline{k}]$.
    Then, $C \vdash\dashv \psi(\overline{k}) \vdash \dashv F(\overline{\gnum{d(k)}})$. But $k = \gnum{\psi}$, so $C = E_k [\overline{k}]$ which is defined to be $d(k)$. So, $C \vdash \dashv F(\overline{\gnum{C}})$.
    
    So, \emph{truth is undefinable}, let $\nat \models \mathrm{True}(\overline{\gnum{\phi}})$ if and only if $\nat \models \phi$. 
    Then, $F(v_1) = \neg \mathrm{True}(v_1)$ so there is $C$ such 
    that $C \models \Dashv \neg \mathrm{True}(\overline{\gnum{C}})\models \Dashv \neg C$.

\subsection*{Recursive Functions}

\begin{definition}
    Primitive recursive functions contain $\underline{\text{zero}}$ 
    and $\underline{\text{succ}}$ and are closed under, projection, composition and primitive recursion.
    Recursive functions are closed under minimilisation as well.
\end{definition}

\begin{proposition}
    Equivalences: $A$ is a decidable set $\iff$ $A$ is $\Delta_1$-definable.\\
    $A$ is a recursively enumerable set $\iff$ $A$ is $\Sigma_1$-definable.
\end{proposition}


\subsection*{Provability Rules}
For $S$ a \emph{provably definable} set of assumptions.
\begin{enumerate}
    \item If $S \proves \phi$ then $\PA \proves \Pr{\phi}$.
    \item $\PA \proves \Pr{\phi \rightarrow \psi} \rightarrow (\Pr{\phi} \rightarrow \Pr{\psi})$.
    \item If $\PA \subseteq S$ then $\PA \proves \Pr{\phi} \rightarrow \Pr{\Pr{\phi}}$
\end{enumerate}

Additionally, $S \proves \phi$ if and only if $\nat \models \Pr{\phi}$.

Let $\PA \subseteq S$ a provably definable set of sentences.
Then, there is a formula $G$, $\nat \models G$ but $S \nvdash G$.
\begin{definition}
    A set $S$ of assumptions is \emph{$n$-inconsistent} if for some $\Sigma_n$ formula $\exists x \psi(x)$, $S \proves \exists x \psi(x)$ but for all $m \in \nat$, $S \proves \neg \psi(\overline{m})$. It is $n$-\emph{consistent} if it is not $n$-inconsistent.
\end{definition}

\begin{definition}
    Formulas $S$ are \emph{$\Sigma_n$-complete} if every $\Sigma_n$ sentence true in $\nat$ is provable from $S$.
\end{definition}

\begin{definition}[Weaker arithmetics]
    $\mathcal{Q}$ is $\PA$ without the induction schema, so it is finitely axiomatisable.
    $\mathcal{R}$ is the collection of all valid sentences of the form $\overline{m} + \overline{n} = \overline{k}$, $\overline{m} \times \overline{n} = \overline{k}$, $\overline{m} \neq \overline{n}$, $\forall v_1(v_1 \leq \overline{n} \rightarrow (v_1 = \overline{0} \lor \ldots \lor \overline{n}))$ and $\forall v_1 (v_1 \leq \overline{n} \lor \overline{n} \leq v_1)$.
\end{definition}

\begin{proposition}
    For every $r \in \mathcal{R}$, $\mathcal{Q} \proves r$.
\end{proposition}

\begin{proposition}
    $\mathcal{R}$ is $\Sigma_0$-complete. Hence, so is $\mathcal{Q}$ and $\PA$.
\end{proposition}

\begin{proposition}
    If $S$ is $\Sigma_0$-complete then it is $\Sigma_1$-complete. Hence, $\mathcal{R}$, $\mathcal{Q}$ and $\PA$ are $\Sigma_1$-complete.
\end{proposition}

\begin{theorem}
    There exists a $\Pi_1$ sentence $G$ such that if $\PA$ is consistent then $\PA \nvdash G$, if $\PA$ is 1-consistent then $\PA \nvdash \neg G$.
\end{theorem}

\end{document}

