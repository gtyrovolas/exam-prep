%!TEX output_directory = .aux
%!TEX copy_output_on_build(true)

\documentclass[a4paper,10pt]{article}
\usepackage[top=2.5cm,bottom=2.5cm,left=2.5cm,right=2.5cm]{geometry}
\usepackage{xcolor,fancyhdr}
\usepackage[utf8]{inputenc}
\usepackage{amsfonts}
\usepackage{amssymb}
\usepackage{amsmath}
\usepackage{mathtools}
\usepackage{amsthm}
\usepackage{tikz}
\usepackage{courier}
\usepackage{stmaryrd}
\include{stylefile}
\usepackage{lineno}
\usepackage{algorithm}
\usepackage{algpseudocode}
\linenumbers
\setpagewiselinenumbers



\makeatother

\title{Category Theory}
\author{Giannis Tyrovolas}
\date{\today}
\theoremstyle{definition}
\newtheorem{theorem}{Theorem}
\newtheorem{prop}[theorem]{Proposition}
\newtheorem*{remark}{Remark}
\DeclarePairedDelimiter\abs{\lvert}{\rvert}
\DeclarePairedDelimiter\norm{\lVert}{\rVert}

\newtheorem{definition}[theorem]{Definition}
\newtheorem{example}[theorem]{Example}
\newtheorem{corollary}[theorem]{Corollary}
\newtheorem{lemma}[theorem]{Lemma}
\newtheorem{proposition}[theorem]{Proposition}
\newtheorem*{idea}{Idea}

\let\vec\mathbf
\let\phi\varphi
\newcommand{\proves}{\vdash}
\newcommand{\zero}{\overline{0}}
\newcommand{\nat}{\mathbb{N}}
\newcommand{\PA}{\mathrm{PA}}
\newcommand{\ProvS}[2][S]{{\mathrm{Pr}_{#1}{(\overline{\gnum{#2}})}}}
\newcommand{\Hom}[3][C]{\mathrm{Hom}_\mathcal{#1}(#2, #3)}
\newcommand{\C}{\mathcal{C}}
\newcommand{\D}{\mathcal{D}}
\newcommand{\id}{\mathrm{id}}
\newcommand{\Set}{\mathrm{Set}}

\begin{document}

%\begin{definition}
%    A \emph{category} $\mathcal{C}$, consists of the following data: 
%    \begin{enumerate}
 %       \item A collection of \emph{objects} $\mathrm{ob}\,\mathcal{C}$,
  %      \item For every two objects $x, y \in \mathrm{ob} \, C$ a collection of \emph{morphisms} $\Hom{x}{y}$.
   %     \item For every $x \in \mathcal{C}$, the identity morphism $\mathrm{id}_x \in \Hom{x}{x}$.
   %     \item A composition map $\circ \colon \Hom{y}{z} \times \Hom{x}{y} \longrightarrow \Hom{x}{z}$ 
   % \end{enumerate}
   % Such that, for all $x, y \in \C$ and $f \in \Hom{x}{y}$: 
   % \[
   %     f \circ \mathrm{id}_x = f \; \; \mathrm{id}_y \circ f = f
   % \]
   % And for all $x, y, z, v$ with $f \in \Hom{x}{y}, g \in \Hom{y}{z}, h \in Hom{z}{v}$:
   % \[
   %     h \circ (g \circ f) = (h \circ g) \circ f
   % \]
%\end{definition}
\begin{definition}
    A \emph{functor} $F \colon \C \longrightarrow \D$ is a map $\mathrm{ob} \C \rightarrow \mathrm{ob} \D$ and a map of morphisms $\Hom[C]{x}{y} \rightarrow \Hom[D]{F(x)}{F(y)}$. Such that $F(\id_x) = \id_{F(x)}$ and $F(g \circ f) = F(g) \circ F(f)$.
\end{definition}

\begin{definition}
    $F \colon \C \rightarrow \D$ is \emph{faithful} if for all $x, y \in \C$, $\Hom{x}{y} \rightarrow \Hom[D]{F(x)}{F(y)}$ is injective.
    It is \emph{full} if every such map is surjective.
    It is \emph{essentially surjective} if for all $d \in \D$ there is $c \in \C$ such that $F(x) \cong d$.
\end{definition}

\begin{definition}
    For two functors $F, G \colon \C \rightarrow \D$, a \emph{natural transformation} $\eta \colon F \Rightarrow D$ is a collection of morphisms $\eta_x \in \Hom[D]{F(x)}{G(x)}$ such that for every $x \xrightarrow{f} y$, $\eta_y \circ F(f) = G(f) \circ \eta_x$. They're natural isos if $\eta_x$ isos.
\end{definition}
\begin{definition}
    Equivalence of categories: $F \colon \C \rightarrow \D$ and $G \colon \D \rightarrow \C$ with natural isomorphisms $e \colon \id_{\C} \Rightarrow GF$, $\epsilon \colon FG \Rightarrow \id_{\D}$. An adjoint equivalence is an equivalence where $F \dashv G$.
\end{definition}

\begin{proposition}
    The following are equivalent: $\C$ and $\D$ are equivalent, $\C$ and $\D$ are adjoint equivalent and there is $F \colon \C \rightarrow \D$ that is fully faithful and essentially surjective. 
\end{proposition}

\begin{definition}
    Two functors $F \colon \C \rightarrow \D$, $G \colon \D \rightarrow \C$ are adjoint if there exist natural transformations $\eta \colon \id_{\C} \Rightarrow GF$ and $\epsilon \colon FG \Rightarrow \id_{\D}$ with $F \xRightarrow{\id_F \circ \eta} FGF \xRightarrow{\epsilon \circ \id_F} F$ and $G \xRightarrow{\eta \circ \id_G} GFG \xRightarrow{\id_G \circ \epsilon} G$.
\end{definition}

The forgetful functor $\mathit{Forget} \colon * \rightarrow \Set$ has a left adjoint $\mathit{Free} \dashv \mathit{Forget}$ for $*$ being Grp, Ab, Vect. $\mathit{Forget} \colon \mathrm{Ab} \rightarrow \mathrm{Grp}$ has a left adjoint, the abelinisation of $G$. For topologies $\mathit{Forget} \colon \mathrm{Top} \longrightarrow \Set$, we have $D \dashv \mathit{Forget} \dashv I$, where $D$ is the discrete topology and $I$ the indiscrete topology.
The forgetful functor from fields has no left adjoint. Such a left adjoint should map $\varnothing$ to an initial object in $\mathrm{Field}$, but fields have no initial object.

\begin{definition}[Comma categories]
    For $F \colon \C \rightarrow \D$ and $G \colon \D \rightarrow \C$, and $x \in \C$ the category $(x \Rightarrow G)$ has objects $(y, x \xrightarrow{f} G(y))$ and morphisms $(y_1, f_1) \rightarrow (y_2, f_2)$ such that $x \xrightarrow{f_1} G(y_1) \rightarrow G(y_2)$ commutes with $f_2$. Similarly $(F \Rightarrow y)$.
\end{definition}

\begin{proposition}
    $F \dashv G$ iff $\Hom[D]{F(x)}{ y} \cong \Hom{x}{G(y)}$ naturally in $x, y$ iff for all $x$, $(F(x), e_x)$ is initialin $(x \Rightarrow G)$.
\end{proposition}
\begin{definition}
    For a locally small category $\C$, the Yoneda Embedding is given by a functor $Y \colon \C \longrightarrow \mathrm{Fun}(\C^\mathrm{op}, \Set)$, where $Y(x) = \Hom{-}{x}$ and $Y(x \xrightarrow{f} y)= (g \mapsto f \circ g)$.
\end{definition}

\begin{definition}
    A functor is representable if it is in the essential image of the Yoneda functor. 
\end{definition}

\begin{lemma}
    The Yoneda lemma states that for any presheaf $F \colon \C^{op} \rightarrow \Set$, the map $\mathrm{Fun}(Y(x), F) \rightarrow F(x)$ given by $\eta \mapsto \eta_x(\id_x)$, is an isomorphism.
\end{lemma}

\begin{proof}
    To construct the inverse, let $f \in F(x)$, then define natural transformation $\epsilon \colon Y(x) \Rightarrow F$ given by $\epsilon_y \colon Y(x)(y) \rightarrow F(y)$ and $g \mapsto F(g)(f)$. Show this is natural by $F$ preserving composition. One inverse is easy, for the other take $\eta$ arbitrary, make diagram with $x \xrightarrow{g} y$ and claim $\eta_y(g) = F(g)(\eta_x(\id_x))$.
\end{proof}

\begin{corollary}
    The Yoneda functor is full and faithful.
\end{corollary}
\begin{proof}
    For $x_1, x_2$, we have that $\Hom[{}]{Y({x_1})}{Y(x_2)} \longrightarrow Y(x_1)(x_2)$ is an isomorphism by Yoneda. 
\end{proof}

\begin{definition}[Representable]
    A presheaf is representable if it is in the essential image of the Yoneda functor.
\end{definition}

\begin{proposition}
    A formal right adjoint to $F \colon \C \rightarrow \D$ is a functor $G^{\mathit{formal}} \colon \mathcal{D} \rightarrow \mathrm{Fun}(\C^{op}, \Set)$ with $y \mapsto (x \mapsto \Hom[D]{F(x)}{y})$. A right adjoint $G$ to $F$ exists if and only if $G^{formal}(y)$ is representable for all $y$.
\end{proposition}

\begin{definition}
    Let $D \colon I \rightarrow \C$. A limit of $D$ is an object $\lim_I D \in \C$ along with maps $f_i \colon \lim_I D \rightarrow D(i)$, such that for every $g \colon i \rightarrow j$, $D(g) \circ f_i = f_j$. It is universal as for any other object $W$ with compatible maps $W \rightarrow D(i)$, there is a unique morphism $W \rightarrow \lim_I F$.
\end{definition}

\begin{proposition}
    The diagonal functor $\Delta \colon \C \rightarrow \mathrm{Fun}(I, \C)$ is  $\Delta(x)(i) = x$. Then $\Hom[\mathrm{Fun}(I, \C)]{\Delta(W)}{F} \cong \Hom{W}{\lim_I F}$, so $\mathrm{colim}_I \dashv \Delta \dashv \lim_I$.
\end{proposition}

\begin{proposition}
    Suppose $\C$ has limits for diagrams of shape $I$ and $J$. Then it has limits of diagrams of shape $I \times J$ and 
    \[
        \lim_{I \times J} F \cong \lim_I \lim_J F \cong \lim_J \lim_I F
    \]
\end{proposition}
For the proof use $\Delta$ as an adjoint.

\begin{theorem}
    $\C$ has limits iff it has products and equalisers. $\C$ has finite limits if it has binary products, final object and equalisers.
\end{theorem}

\begin{proof}
    For $F \colon I \rightarrow \C$, for every morphism $f \colon i \rightarrow j$, let $\prod_{k \in I} F(k) \rightarrow F(j)$ the projection map and the composite map $\prod_{k \in I} F(k) \rightarrow F(i) \xrightarrow{F(f)} F(j)$. Then, by the universal property of the product we get unique maps: 

    \[
        \prod_{k \in I} F(k) \rightrightarrows \prod_{(i \rightarrow j) \in \mathrm{Fun}([1], J)} F(j)
    \]
    By the equalisers, we get $E$ the limit of $F$.
\end{proof}

\begin{definition}
    A morphism $f \colon X \rightarrow Y$ is a \emph{monomorphism} if for every $g, h \colon Y \rightarrow Z$, $f \circ g = f \circ h$ implies $g = h$. For an epimorphism $f$, $g \circ f = h \circ f$ implies $g = h$. 
\end{definition}
\begin{definition}
    Equalisers are regular monomorphisms, coequalisers are regular epimorphisms.
\end{definition}

\begin{center}
\begin{tabular}{ |c|c|c|c|}
    \hline
    
    \multicolumn{2}{|c|}{Limits} & \multicolumn{2}{|c|}{Colimits}\\
    \hline
    \textbf{Final}& $\Set: \{1\}, \mathrm{Grp}: \{e\}$ & \textbf{Initial} & $\Set: \varnothing, \mathrm{Grp}: \{e\}$ Fields: None\\
    \hline
    \textbf{Products} & $\times$ in Grp, Set, Vect & \textbf{Co-products} & Set: $\sqcup$, Grp: free product, Ab: $\times$ \\
    \hline
    \textbf{Equal} &Set:$x$ with $f(x) = g(x)$, & \textbf{Coeq} & $\Set$: $Y / f(x) \sim g(x)$, Grp: $Y/S$ $f(x)g(x)^{-1}$ \\
    \hline
    \textbf{Pullback} &$\{(x, y) \mid f(x) = g(y)\}$ & \textbf{Pushout}& \\
    \hline
\end{tabular}
\end{center}

\begin{theorem}
    Let $F \dashv G$ then $F$ preserves colimits and $G$ preserves limits.
\end{theorem}

\begin{theorem}
    $F \colon \C \rightarrow \D$ for $\C, \D$ locally small, and $\C$ has small colimits.
    $F \dashv G$ iff $F$ preserves colimits and for all $x$, $(F \Rightarrow x)$ the solution set holds. A category $\C$ satisfied it if:
    there is $I$ small, $\{c_i\}_{i \in I}$ such that for each $x \in \C$ there is $c_i$ with $\Hom{x}{c_i}$ non-empty.
\end{theorem}


\begin{definition}[Monads]
    A monad $T \colon \C \rightarrow \C$ is a functor with unit $\eta \colon \id_\C \Rightarrow T$ and multiplication $\mu \colon T^2 \Rightarrow T$, satisfying: $T^3 \xRightarrow{id_T \circ \mu} T^2 \xRightarrow{\mu} T$ is equal to $T^3 \xRightarrow{\mu \circ \id_T} T^2 \xRightarrow{\mu} T$ and $T \xRightarrow{\id_T \circ \eta} T^2 \xRightarrow{\mu} T$ and $T \xRightarrow{\eta \circ \id_T} T^2 \xRightarrow{\mu} T$ are both equal to the identity.
\end{definition}

\begin{proposition}
    Let $F \dashv G$ with $F \colon \C \rightarrow \D$, then $GF$ is a monad in $\C$ and $FG$ is a comonad in $\D$.
    For the proof, let $\eta \colon \id_\C \Rightarrow GF$  and $\mu \colon GFGF \xRightarrow{\id_G \circ \epsilon \circ \id_F} GF$. Diagrams for $T^2$ and then append $G$ and $F$.
\end{proposition}

\begin{definition}
    For a monad $T$ an algebra $\mathrm{Alg}_T(\C)$ is the category with objects $(x, T(x) \xrightarrow {\alpha_x} x)$ for $x \in \C$ with $\alpha_x$ such that $T^2(x) \xrightarrow{\mu_x} T(x) \xrightarrow{\alpha_x} x$ is equal to $T^2(x) \xrightarrow{T(\alpha_x)} T(x) \xrightarrow{\alpha_x} x$ and $x \xrightarrow{\eta_x} T(x) \xrightarrow{\alpha_x} x$ is the identity. The morphisms $(x, \alpha_x) \rightarrow (y, \alpha_y)$ are given by $f \colon x \rightarrow y$ such that $T(x) \xrightarrow{\alpha_x} x \xrightarrow{f} y$ and $T(x) \xrightarrow{T(f)} T(y) \xrightarrow{\alpha_y} y$ commutes.
\end{definition}

\begin{proposition}
    The forgetful functor $F \colon \mathrm{Alg}_T(\C) \rightarrow \C$ has a left adjoint $L$ and $FL = T$.
    We have $L(x) = (T(x), \mu_x)$. Write down the commutative square for a morphism $T(x) \xrightarrow{f} A$ and then $T(x) \xrightarrow{T(\eta_x)} T^2(x) \xrightarrow{\mu_x} T(x)$ commutes. So $f$ is uniquely determined by $x \xrightarrow{\eta_x} Tx \xrightarrow{f} A$ giving an isomorphism of Homs.
\end{proposition}

\begin{definition}
    For $F \dashv G$ let $G_{\mathit{enh}} \colon \D \rightarrow \mathrm{Alg}_T(\C)$ with $G_{\mathit{enh}}(x) = (G(x), GFG(x) \xrightarrow{G(\epsilon_x)} G(x))$. $G$ is \emph{monadic} if there is $F \dashv G$ and for $T = GF$, $G_\mathit{enh} \colon \D \rightarrow \mathrm{Alg}_T(\C)$ is an equivalence.
\end{definition}

\begin{definition}
    A functor $G \colon \C \rightarrow \D$ is \emph{conservative} if when $G(f)$ is an isomorphism so is $f$.
    The forgetful functors from Ab, Grp, Vect are conservative, the one from Top is not.
\end{definition}

\begin{definition}
    A fork is a cocone $x \rightrightarrows y  \xrightarrow{e} z $ so that $e \circ g = e \circ f$. It is split if there are $s \colon z \rightarrow y$, $t \colon y \rightarrow x$ such that $es = \id_z$, $ft = \id_y$ and $gt = se$.
    Split forks are coequalisers.
\end{definition}

\begin{definition}
    Morphisms $f, g \colon x \rightarrow y$ are a split pair if their coequaliser exists and is split. They are $G$ split if $G(f), G(g)$ is split.
\end{definition}

\begin{theorem}
    A functor $G \colon \D \rightarrow \C$ is \emph{monadic} if and only if: it has a left adjoint, it is conservative and every $G$-split pair admits a coequaliser in $\D$ and is preserved by $G$.
\end{theorem}
\end{document}

