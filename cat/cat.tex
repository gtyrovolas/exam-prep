%!TEX output_directory = .aux
%!TEX copy_output_on_build(true)

\documentclass[a4paper,10pt]{article}
\usepackage[top=2.5cm,bottom=2.5cm,left=2.5cm,right=2.5cm]{geometry}
\usepackage{xcolor,fancyhdr}
\usepackage[utf8]{inputenc}
\usepackage{amsfonts}
\usepackage{amssymb}
\usepackage{amsmath}
\usepackage{mathtools}
\usepackage{amsthm}
\usepackage{tikz}
\usepackage{courier}
\usepackage{stmaryrd}
\include{stylefile}
\usepackage{lineno}
\usepackage{algorithm}
\usepackage{algpseudocode}
\linenumbers
\setpagewiselinenumbers



\makeatother

\title{Category Theory}
\author{Giannis Tyrovolas}
\date{\today}
\theoremstyle{definition}
\newtheorem{theorem}{Theorem}
\newtheorem{prop}[theorem]{Proposition}
\newtheorem*{remark}{Remark}
\DeclarePairedDelimiter\abs{\lvert}{\rvert}
\DeclarePairedDelimiter\norm{\lVert}{\rVert}

\newtheorem{definition}[theorem]{Definition}
\newtheorem{example}[theorem]{Example}
\newtheorem{corollary}[theorem]{Corollary}
\newtheorem{lemma}[theorem]{Lemma}
\newtheorem{proposition}[theorem]{Proposition}
\newtheorem*{idea}{Idea}

\let\vec\mathbf
\let\phi\varphi
\newcommand{\proves}{\vdash}
\newcommand{\zero}{\overline{0}}
\newcommand{\nat}{\mathbb{N}}
\newcommand{\PA}{\mathrm{PA}}
\newcommand{\ProvS}[2][S]{{\mathrm{Pr}_{#1}{(\overline{\gnum{#2}})}}}
\newcommand{\Hom}[3][C]{\mathrm{Hom}_\mathcal{#1}(#2, #3)}
\newcommand{\C}{\mathcal{C}}
\newcommand{\D}{\mathcal{D}}
\newcommand{\id}{\mathrm{id}}

\begin{document}

\begin{definition}
    A \emph{category} $\mathcal{C}$, consists of the following data: 
    \begin{enumerate}
        \item A collection of \emph{objects} $\mathrm{ob}\,\mathcal{C}$,
        \item For every two objects $x, y \in \mathrm{ob} \, C$ a collection of \emph{morphisms} $\Hom{x}{y}$.
        \item For every $x \in \mathcal{C}$, the identity morphism $\mathrm{id}_x \in \Hom{x}{x}$.
        \item A composition map $\circ \colon \Hom{y}{z} \times \Hom{x}{y} \longrightarrow \Hom{x}{z}$ 
    \end{enumerate}
    Such that, for all $x, y \in \C$ and $f \in \Hom{x}{y}$: 
    \[
        f \circ \mathrm{id}_x = f \; \; \mathrm{id}_y \circ f = f
    \]
    And for all $x, y, z, v$ with $f \in \Hom{x}{y}, g \in \Hom{y}{z}, h \in Hom{z}{v}$:
    \[
        h \circ (g \circ f) = (h \circ g) \circ f
    \]
\end{definition}

\begin{definition}
    A \emph{functor} $F \colon \C \longrightarrow \D$ is a map $\mathrm{ob} \C \rightarrow \mathrm{ob} \D$ and a map of morphisms $\Hom[C]{x}{y} \rightarrow \Hom[D]{F(x)}{F(y)}$. Such that $F(\id_x) = \id_{F(x)}$ and $F(g \circ f) = F(g) \circ F(f)$.
\end{definition}

\begin{definition}
    $F \colon \C \rightarrow \D$ is \emph{faithful} if for all $x, y \in \C$, $\Hom{x}{y} \rightarrow \Hom[D]{F(x)}{F(y)}$ is injective.
    It is \emph{full} if every such map is surjective.
\end{definition}

\begin{definition}
    A functor $F \colon \C \rightarrow \D$ is essentially surjective if for all $d \in \D$ there is $c \in \C$ such that $F(x) \cong d$.
\end{definition}

\begin{definition}
    For two functors $F, G \colon \C \rightarrow \D$, a \emph{natural transformation} $\eta \colon F \Rightarrow D$ is a collection of morphisms $\eta_x \in \Hom[D]{F(x)}{G(x)}$ such that for every $x \xrightarrow{f} y$, $\eta_y \circ F(f) = G(f) \circ \eta_x$. 
\end{definition}
It is a natural isomorphism if all morphisms $\eta_x$ are isomorphisms.

\begin{definition}
    Equivalence of categories: $F \colon \C \rightarrow \D$ and $G \colon \D \rightarrow \C$ with natural isomorphisms $e \colon \id_{\C} \Rightarrow GF$, $\epsilon \colon FG \Rightarrow \id_{\D}$. An adjoint equivalence is an equivalence where $F \dashv G$.
\end{definition}

\begin{proposition}
    The following are equivalent: $\C$ and $\D$ are equivalent, $\C$ and $\D$ are adjoint equivalent and there is $F \colon \C \rightarrow \D$ that is fully faithful and essentially surjective. 
\end{proposition}




\end{document}

