%!TEX output_directory = .aux
%!TEX copy_output_on_build(true)

\documentclass[a4paper,10pt]{article}
\usepackage[top=2.5cm,bottom=2.5cm,left=2.5cm,right=2.5cm]{geometry}
\usepackage{xcolor,fancyhdr}
\usepackage[utf8]{inputenc}
\usepackage{amsfonts}
\usepackage{amssymb}
\usepackage{amsmath}
\usepackage{mathtools}
\usepackage{amsthm}
\usepackage{tikz}
\usepackage{stmaryrd}
\include{stylefile}
\usepackage{lineno}
\linenumbers
\setpagewiselinenumbers

\title{Metric Spaces and Complex Analysis}
\author{Giannis Tyrovolas}
\date{\today}
\theoremstyle{definition}
\newtheorem{theorem}{Theorem}
\newtheorem{prop}[theorem]{Proposition}
\newtheorem*{remark}{Remark}
\DeclarePairedDelimiter\abs{\lvert}{\rvert}
\DeclarePairedDelimiter\norm{\lVert}{\rVert}

\newtheorem{definition}[theorem]{Definition}
\newtheorem{example}[theorem]{Example}
\newtheorem{corollary}[theorem]{Corollary}
\newtheorem{lemma}[theorem]{Lemma}
\newtheorem{proposition}[theorem]{Proposition}
\newtheorem*{idea}{Idea}

\let\vec\mathbf
\let\phi\varphi

\begin{document}

\subsection*{Types}

\begin{definition}
    For a theory $T$, and variables $\vec{x}$, a \emph{partial type} $P$ is a set of formulas where $T \cup P$ is consistent.
\end{definition}

\begin{example}
    For $T = \text{Th}(\left\langle \mathbb{Z}, +, -, 0, 1\right\rangle )$, $P(x) = \{\exists y(y + y \ldots +y = x)\} \cup \{x \neq 0\}$, is a partial type. This can be proven by compactness.
\end{example}

\begin{definition}
    For a theory $T$, a type $P$ is principal if for some $\theta(\vec{x})$, $T\cup \theta(\vec{x}) \models P$ and $T \cup \theta$ is consistent.
\end{definition} 

\begin{theorem}
    If $P$ is not principal it is omitted in some model of $T$. If $P$ is principal and $T$ is complete then every model of $T$ realises $P$.
\end{theorem}

\subsection*{Embeddings}

\begin{theorem}
    If $A \leqslant B$ then for every \emph{quantifier free} $\varphi(x_1, \ldots x_n)$, 
    \[\varphi ^ {\underline{B}} \cap A^k = \varphi ^ {\underline{A}}.\]
    If $A \preceq B$ then this is true for all formulas $\varphi$.
\end{theorem}

\subsection*{Preservation Theorems}

\begin{theorem}
    For a theory $T$, $\underline{A} \models T_{\forall}$ if and only if there exists $\underline{B} \models T$ with $\underline{A}  \leqslant \underline{B}$. 
\end{theorem}

\begin{corollary}
    The theory of fields is not universal as, $\underline{Z} \leqslant \underline{Q}$ but $\underline{Q}$ is a field and $\underline{Z}$ is not.
\end{corollary}

\begin{theorem}
    Sentence $\sigma$ is universal if and only if for all $B \models \sigma$ and $A \leqslant B$, $A \models \sigma$.
\end{theorem}

\begin{example}
    For $F$ the theory of fields, $F_\forall$ is the theory of integral domains.
    That is because every integral domain can be embedded in a field.
\end{example}

\begin{theorem}
    For a chain $\underline{A_1} \leqslant \underline{A_2} \leqslant \ldots$, let $\underline{A^*}$ be the limit of the chain. Then every AE sentence $\sigma$ which holds for all $\underline{A_i}$, holds for $\underline{A^*}$.
\end{theorem}

\subsection*{Quantifier elimination}

\begin{definition}
    Theory $T$ admits quantifier elimination if for any formula $\theta(\vec{x})$, there exists a quantifier free formula $\tilde{\theta}(\vec{x})$ such that:
    \[
        T \models \forall \vec{x} (\theta \leftrightarrow \tilde{\theta})
    \]
\end{definition}

\begin{theorem}
    If $L$ has no constant or function symbols and $T$ admits Q.E. then $T$ is complete.
\end{theorem}

\begin{example}
    \begin{itemize}
        \item Th$(\left\langle \mathbb{Q}, < \right\rangle )$ admits QE and so is complete.
        \item ACF admits QE. But, the only thing ACF does not decide is the field characteristic. Hence, ACF$_p$ for $p$ prime or zero is complete.
        \item Th$(\left\langle \mathbb{R}, +, -, \times, 0, 1\right\rangle )$ does not admit Q.E. Atomic sentences with one variable define only, finite and cofinite sets. But $\varphi(x) = \exists y (y^2 = x)$ defines the positive numbers.
        \item Th$(\left\langle \mathbb{R}, +, -, \times, 0, 1, < \right\rangle )$ admits Q.E. by Tarski. It is complete because the order is complete and so determines equality.
    \end{itemize}
\end{example}

\begin{remark}
    If $T$ admits Q.E. and $\underline{A_1},\underline{A_2} \models T$ and $ \underline{A_1} \leqslant \underline{A_2}$ then $\underline{A_1} \preceq \underline{A_2}$.
\end{remark}

\begin{theorem}
    If it exists, there is only one way to extend a universal theory to a Q.E. theory. Prove by taking $S \models \underline{A_1}$ and $\underline{A_1} \leqslant \underline{B_1} \models T$ and build chains. The limits are equal and $\underline{A_1} \preceq A_2 \preceq C$.
\end{theorem}

\begin{theorem}[Equivalence]
    \begin{enumerate}
        \item $T$ has Q.E.
        \item Any partial isomorphism between models of $T$ is elementary. It is enough to consider isomorphisms on finitely generated subsets.
        \item For any $\mathcal{M} \models T$ and any $\vec{a} \in \mathcal{M}^n$, $T \cup \mathrm{diag}(\vec{a})$ is complete.
    \end{enumerate}
\end{theorem}

\subsection*{Categoricity}

\begin{definition}
    A theory $T$ for a cardinal $\kappa$ is $\kappa$-categorical if there exist models $\underline{A}, \underline{B} \models T$ and for $\abs{A} = \abs{B} = \kappa$, $A \cong B$.
\end{definition}

\begin{proposition}
    If $T$ has no finite models, and for $\kappa \geqslant \abs{L} + \aleph_0$ $T$ is $\kappa$-categorical, then $T$ is complete.
\end{proposition}


\begin{example}
    \begin{enumerate}
        \item Theory of equality $T_=$ is categorical for every cardinal. So $T_{\infty}$ is complete.
        \item $\mathrm{Vect}_K$ is categorical for every $\kappa \geqslant \abs{K}$, so $\mathrm{Vect}_K \cup T_\infty$ is complete.
        \item $\mathrm{DLO}$ is $\aleph_0$-categorical and has no finite models. Proof by back and forth lemma.
    \end{enumerate}
\end{example}

\begin{definition}[Atomic Model]
    $\underline{A}$ is an \emph{atomic} model of a \emph{complete} theory $T$ if for any $\vec{a} \in A^n$ there is $\phi(\vec{x})$ such that $\underline{A} \models \phi(\vec{a})$ and for any $\psi(\vec{x})$:
    \[
        T \vDash \forall x (\phi \rightarrow \psi) \text{ or } T \vDash \forall x (\phi \rightarrow \neg \psi)
    \]
\end{definition}

\begin{definition}
    A model $\underline{A} \models T$ is \emph{homogeneous} if for any $\vec{a}, \vec{b} \in A^n$ that satisfy the same formulas, there is an automorphism $\alpha \colon A \longrightarrow A$ such that $\alpha(a_i) = b_i$.
\end{definition}

\begin{definition}
    A model $A \models T$ is \emph{prime} if for any model $B \models T$, $A$ embeds elementarily to $B$.
\end{definition}

\begin{proposition} 
    \emph{Countable} atomic models are isomorphic. In fact, every finite partial isomorphism can be extended to an isomorphism. They are also prime and homogeneous.
\end{proposition}

\begin{definition}[Type]
    The \emph{$n$-type} of an $n$-tuple $\vec{a} \in A^n$ is the set of formulas satisfied by $\vec{a}$, denoted by $\mathrm{tp}_A(\vec{a})$. In particular, it is a partial type for the $\mathrm{Th}(\underline{A})$. It is complete as $\phi(\vec{x}) \in \mathrm{tp}_A(\vec{a})$ or $\neg \phi(\vec{x})\in \mathrm{tp}_A(\vec{a})$.    
\end{definition}

\begin{proposition}
    For a complete theory $T$ the atomic models realise the fewest types.
\end{proposition}

\begin{proposition}
    For a countable language $L$, Prime $\iff$ Countable and Atomic.
\end{proposition}

\begin{corollary}
    The prime models of $T$ are isomorphic, by uniqueness of countable \& atomic.
\end{corollary}

\begin{proposition}
    If for each $n$ the set of $n$-types is countable, then $T$ has a prime model.
\end{proposition}

\begin{definition}
    A countable model $\mathcal{M} \models T$ is \emph{universal}, if every countable model embeds elementarily into $\mathcal{M}$.
\end{definition}

\begin{theorem}[Ryll-Nardzewski]
    Let $T$ complete and $L$-countable. Then, $T$ is $\aleph_0$-categorical $\iff$ every countable model is prime $\iff$ every countable model is atomic $\iff$ every type is principal $\iff$ there are only finitely many $n$-types $\iff$ $n$-formulas $\phi(\vec{x})$ up to $T$ equivalence is finite $\iff$ \\every countable model is universal $\iff$ a countable model is prime and universal $\iff$ \\ every countable model is universal and homogeneous.
\end{theorem}

\begin{definition}
    A \emph{saturated} model is a model that realises all $n$-types and is homogeneous. Equivalently: \\
    If $\mathcal{M}$ is saturated, for all $B \subseteq \mathcal{M}$ and $\abs{B} < \abs{\mathcal{M}}$, $\mathcal{M}_B$ realises all 1-types of $\mathrm{Th}(\mathcal{M}_B)$.
\end{definition}

\begin{definition}
    A group $G$ applied to a $G$-set is oligomorphic if there are finitely many orbits of $G$.
\end{definition}

\end{document}

