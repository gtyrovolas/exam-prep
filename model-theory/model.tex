%!TEX output_directory = .aux
%!TEX copy_output_on_build(true)

\documentclass[a4paper,10pt]{article}
\usepackage[top=2.5cm,bottom=2.5cm,left=2.5cm,right=2.5cm, showframe]{geometry}
\usepackage{xcolor,fancyhdr}
\usepackage[utf8]{inputenc}
\usepackage{amsfonts}
\usepackage{amssymb}
\usepackage{amsmath}
\usepackage{mathtools}
\usepackage{amsthm}
\usepackage{tikz}
\usepackage{stmaryrd}
\include{stylefile}
\usepackage{lineno}
\linenumbers
\setpagewiselinenumbers

\title{Metric Spaces and Complex Analysis}
\author{Giannis Tyrovolas}
\date{\today}
\theoremstyle{definition}
\newtheorem{theorem}{Theorem}
\newtheorem{prop}[theorem]{Proposition}
\newtheorem*{remark}{Remark}
\DeclarePairedDelimiter\abs{\lvert}{\rvert}
\DeclarePairedDelimiter\norm{\lVert}{\rVert}

\newtheorem{definition}[theorem]{Definition}
\newtheorem{example}[theorem]{Example}
\newtheorem{corollary}[theorem]{Corollary}
\newtheorem{lemma}[theorem]{Lemma}
\newtheorem{proposition}[theorem]{Proposition}
\newtheorem*{idea}{Idea}

\let\vec\mathbf
\let\phi\varphi
\let\preceq\preccurlyeq
\let\leq\leqslant
\let\geq\geqslant
\let\implies\Rightarrow

\begin{document}

\begin{theorem}[Upwards Lowenheim-Skolem] For an infinite $L$-structure $\underline{A}$ and $\kappa \geq \abs{A} + \abs{L}$ there is an $L$-structure $\underline{B}$ such that $\underline{A} \preceq \underline{B}$.
\end{theorem}

\begin{proof}
    For $\abs{M} = \kappa$, expand $L$ with constants, $L_{A,M}$. $\Sigma = \mathrm{CDiag}(\underline{A}) \cup \{c_i \neq c_j \in M\}$ is satisfiable. 
\end{proof}

\begin{theorem}[Downwards Lowenheim-Skolem]
    For $\underline{B}$ an $L$-structure, $S \subseteq B$, there exists $\underline{A}$ such that $S \subseteq A$, $\abs{A} \leq \max{(\abs{S}, \abs{L})}$ and $\underline{A} \preceq \underline{B}$.

\end{theorem}

\begin{definition}
    For a theory $T$, and variables $\vec{x}$, a \emph{partial type} $P$ is a set of formulas where $T \cup P$ is consistent.
\end{definition}

\begin{example}
    For $T = \text{Th}(\left\langle \mathbb{Z}, +, -, 0, 1\right\rangle )$, $P(x) = \{\exists y(y + y \ldots +y = x)\} \cup \{x \neq 0\}$, is a partial type. This can be proven by compactness, first add constants.
\end{example}

\begin{definition}
    For a theory $T$, a type $P$ is principal if for some $\theta(\vec{x})$, $T\cup \theta(\vec{x}) \models P$ and $T \cup \theta$ is consistent.
\end{definition} 

\begin{theorem}[Omitting types]
    Let $T$ complete and $L$ countable. Let $P$ a countable set of non-principal types.
    Then, there is a countable model of $T$ omitting every type in $P$.
\end{theorem}
\begin{proof}
    To construct the model, expand the language with countable constants, enumerate the sentences, formulas and closed tuples. At $T_{3n+1}$ add $\sigma_n$ or $\neg \sigma_n$ depending on consistency, add $\neg \exists y \phi_n(y)$ or $\phi_n(c_k)$ add $\psi(\vec{x}) \in P$ such that $T_{3n+2} \cup \neg \psi(\vec{t}_n)$ is consistent. Such a $\psi$ exists because adding constant doesn't un-principal a type and for $\theta = \sigma_n \land (x = c)$, $T \cup \theta(\vec{x}) \nvDash  \phi(\vec{x})$, so $T \cup \{\sigma, \neg \phi(c)\} $.
\end{proof}

\begin{theorem}
    If $A \leqslant B$ then for every \emph{quantifier free} $\varphi(x_1, \ldots x_n)$, $\varphi ^ {\underline{B}} \cap A^k = \varphi ^ {\underline{A}}$.
    If $A \preceq B$ then this is true for all formulas $\varphi$.
\end{theorem}

\begin{proposition}[Tarski-Vaught criterion]
    If $\underline{A} \leqslant \underline{B}$ and for $\phi(\vec{x}, y )$ and $\vec{a} \in A^n$, $\underline{B} \models \phi(\vec{a}, d)$ for $d \in B$ then $\underline{B} \models \phi(\vec{a}, c)$ for $c \in A$, then $\underline{A} \preceq \underline{B}$.
\end{proposition}

\begin{definition}
    For $\underline{A}$ let $L_A = L \cup \{c_a \mid a \in A\}$. $\underline{A_A}$ is an $L_A$-structure. The \emph{diagram} $\mathrm{Diag}(\underline{A})$ is all q.f. $L_A$ sentences true in $A_A$.
\end{definition}

\begin{theorem}
    There is a 1-1 correspondence between models of $\mathrm{Diag}(\underline{A}) \cup T$ and pairs $(\underline{B}, \underline{A})$ where $\underline{B} \models T$ and $\underline{A} \leq \underline{B}$.
\end{theorem}

\begin{proof}
    Let $\underline{C} \models \mathrm{Diag}(A) \cup T$, $\underline{B} = \underline{C}_{\mid L}$, so $\underline{B} \models T$, build $f \colon A \rightarrow B$, $a \mapsto c_a^{\underline{C}}$. Then $f$ is an embedding as for q.f. $\phi$, $\underline{A} \models \phi(\vec{a})$, $\phi(\vec{c}_a) \in \mathrm{Diag}(\underline{A}) \implies \underline{C} \models  \phi(\vec{c}_a) \implies \underline{C} \models \phi(\vec{c}_a^C) \implies \underline{C} \models \phi(f(\vec{a})) \implies \underline{B} \models \phi(f(\vec{a}))$. If $A \nvDash \phi(\vec{a}) \implies A \models \neg \phi(\vec{a}) \implies B \models \neg \phi(\vec{a}) \implies B \nvDash \phi(\vec{a})$.
\end{proof}



\begin{theorem}
    For a theory $T$, $\underline{A} \models T_{\forall}$ if and only if there exists $\underline{B} \models T$ with $\underline{A}  \leqslant \underline{B}$. 
\end{theorem}

\begin{proof}
    $(\Rightarrow)$ There is $\underline{A} \leq \underline{B}$ iff $\underline{B} \models \mathrm{Diag}(\underline{A}) \cup T$. iff finitely satisfiable iff $T + \phi$ for $\phi \in \mathrm{Diag}(\underline{A})$ is satisfiable iff $T \nvDash \neg \phi(c_1, c_2, \ldots, c_n)$ iff $T \nvDash \forall \vec{x} \neg \phi(\vec{x}) $. But $A \models \exists \vec{x} \phi(\vec{x})$ so $\forall \vec{x} \neg \phi(\vec{x}) \notin T_\forall$.
\end{proof}


\begin{corollary}
    The theory of fields is not universal as, $\underline{Z} \leqslant \underline{Q}$ but $\underline{Q}$ is a field and $\underline{Z}$ is not.
\end{corollary}

\begin{theorem}
    Sentence $\sigma$ is universal if and only if for all $B \models \sigma$ and $A \leqslant B$, $A \models \sigma$.
\end{theorem}

\begin{example}
    For $F$ the theory of fields, $F_\forall$ is the theory of integral domains.
    That is because every integral domain can be embedded in a field.
\end{example}

\begin{theorem}
    For a chain $\underline{A_1} \leqslant \underline{A_2} \leqslant \ldots$, let $\underline{A^*}$ be the limit of the chain. Then every AE sentence $\sigma$ which holds for all $\underline{A_i}$, holds for $\underline{A^*}$.
\end{theorem}

\begin{definition}
    Theory $T$ admits quantifier elimination if for any formula $\theta(\vec{x})$, there exists a quantifier free formula $\tilde{\theta}(\vec{x})$ such that:$
        T \models \forall \vec{x} (\theta \leftrightarrow \tilde{\theta})
    $
\end{definition}

\begin{theorem}
    If $L$ has no constant or function symbols and $T$ admits Q.E. then $T$ is complete.
\end{theorem}

\begin{example}
     Th$(\left\langle \mathbb{Q}, < \right\rangle )$ admits QE and so is complete. ACF admits QE. But, the only thing ACF does not decide is the field characteristic. Hence, ACF$_p$ for $p$ prime or zero is complete. Th$(\left\langle \mathbb{R}, +, -, \times, 0, 1\right\rangle )$ does not admit Q.E. Atomic sentences with one variable define only, finite and cofinite sets. But $\varphi(x) = \exists y (y^2 = x)$ defines the positive numbers. Th$(\left\langle \mathbb{R}, +, -, \times, 0, 1, < \right\rangle )$ admits Q.E. by Tarski. It is complete because the order is complete and so determines equality.
\end{example}

\begin{remark}
    If $T$ admits Q.E. and $\underline{A_1},\underline{A_2} \models T$ and $ \underline{A_1} \leqslant \underline{A_2}$ then $\underline{A_1} \preceq \underline{A_2}$.
\end{remark}

\begin{theorem}
    If it exists, there is only one way to extend a universal theory to a Q.E. theory. Prove by taking $\underline{A_1} \models S$ and $\underline{A_1} \leqslant \underline{B_1} \models T$ and build chains. The limits are equal and $\underline{A_1} \preceq A_2 \preceq C$.
\end{theorem}

\begin{theorem}[Equivalence]
    $T$ has Q.E.\\ Any partial isomorphism between models of $T$ is elementary. It is enough to consider isomorphisms on finitely generated subsets.
    \\ For any $\mathcal{M} \models T$ and any $\vec{a} \in \mathcal{M}^n$, $T \cup \mathrm{diag}(\vec{a})$ is complete.
\end{theorem}

\begin{proof}
    $(1) \implies (2)$, $\mathcal{M} \models \phi(\vec{a}) \implies \mathcal{M} \models \tilde{\phi}(\vec{a}) \implies \mathcal{N} \models \tilde{\phi}(f(\vec{a}))$. \\
    $(2) \implies (1)$, pick $\mathcal{M} \models T \cup \mathrm{Diag}(A) \cup \{\phi(\vec{c}\}$ and $\mathcal{N} \models T \cup \mathrm{Diag}(A) \cup \{ \neg\phi(\vec{c}\}$ then $\tau(\vec{c})^\mathcal{M} \mapsto \tau(\vec{c}^\mathcal{N})$ is a non-elementary partial isomorphism. \\ 
    $(3) \implies (1)$ For $\phi$ take $\underline{qfco}(\phi)$ all q.f. s.t. $\phi \iff \theta$. Want to show: $T \cup \underline{qfco}(\phi) \models \phi$. Now, if $T \cup \mathrm{Diag}(A) \models \neg \phi(a)$, then $T \cup \{\psi\} \models \neg \phi$ for $\psi$ quantifier free.
\end{proof}

\begin{definition}
    A theory $T$ for a cardinal $\kappa$ is $\kappa$-\emph{categorical} if there exist models $\underline{A}, \underline{B} \models T$ with $\abs{A} = \abs{B} = \kappa$ and this implies $A \cong B$.
\end{definition}

\begin{proposition}[Los-Vaught Test]
    If $T$ has no finite models, and for $\kappa \geqslant \abs{L} + \aleph_0$, $T$ is $\kappa$-categorical, then $T$ is complete.
\end{proposition}

\begin{proof}
    Take $\mathcal{M} \models T$, $\abs{\mathcal{M}} = \kappa$. Then, for any sentence $\sigma$, $\mathcal{M} \models \sigma$ or $\mathcal{M} \models \neg \sigma$, wlog let it be $\sigma$. Then, $T \cup \{\neg \sigma\}$ has no model of cardinality $\kappa$, by the Lowenheim-Skolems $T \cup \{\neg \sigma\}$ has no infinite models.
\end{proof}


\begin{example}
        Theory of equality $T_=$ is categorical for every cardinal. So $T_{\infty}$ is complete.
         $\mathrm{Vect}_K$ is categorical for every $\kappa > \abs{K}$, so $\mathrm{Vect}_K \cup T_\infty$ is complete. But, $\mathrm{Vect}_\mathbb{Q}$ is not $\aleph_0$-categorical.
         $\mathrm{DLO}$ is $\aleph_0$-categorical and has no finite models. Proof by back and forth lemma. It is not $\aleph_1$-categorical, take $\mathbb{R} \sqcup \mathbb{Q} \ncong \mathbb{R}$.
\end{example}

\begin{definition}[Atomic Model]
    $\underline{A}$ is an \emph{atomic} model of a \emph{complete} theory $T$ if for any $\vec{a} \in A^n$ there is $\phi(\vec{x})$ such that $\underline{A} \models \phi(\vec{a})$ and for any $\psi(\vec{x})$:
$        T \vDash \forall x (\phi \rightarrow \psi) \text{ or } T \vDash \forall x (\phi \rightarrow \neg \psi)$
\end{definition}

\begin{definition}
    A model $\underline{A} \models T$ is \emph{homogeneous} if for any $\vec{a}, \vec{b} \in A^n$ that satisfy the same formulas, there is an automorphism $\alpha \colon A \longrightarrow A$ such that $\alpha(a_i) = b_i$.
\end{definition}

\begin{definition}
    A model $A \models T$ is \emph{prime} if for any model $B \models T$, $A$ embeds elementarily to $B$.
\end{definition}

\begin{proposition} 
    \emph{Countable} atomic models are isomorphic. In fact, every finite partial isomorphism can be extended to an isomorphism. They are also prime and homogeneous. For the one step lemma, $(c_1, \ldots, c_n, a)$, $\phi(\vec{c}, a)$ the type,  $\underline{A} \models \exists y \phi(\vec{c}, y) \implies \underline{B} \models \exists y \phi(f(\vec{c}))$. Then check elementary.
\end{proposition}

\begin{definition}[Type]
    The \emph{$n$-type} of an $n$-tuple $\vec{a} \in A^n$ is the set of formulas satisfied by $\vec{a}$, denoted by $\mathrm{tp}_A(\vec{a})$. $\mathrm{tp}_A(\vec{a})$ is a partial type for the $\mathrm{Th}(\underline{A})$. It is complete as $\phi(\vec{x}) \in \mathrm{tp}_A(\vec{a})$ or $\neg \phi(\vec{x})\in \mathrm{tp}_A(\vec{a})$.    
\end{definition}

\begin{proposition}
    For a complete theory $T$ the atomic models realise the fewest types.
\end{proposition}

\begin{proposition}
    For a countable language $L$, Prime $\iff$ Countable and Atomic.
\end{proposition}

\begin{corollary}
    The prime models of $T$ are isomorphic, by uniqueness of countable \& atomic.
\end{corollary}

\begin{proposition}
    If for each $n$ the set of $n$-types is countable, then $T$ has a prime model.
\end{proposition}

\begin{definition}
    Countable $\mathcal{M} \models T$ is \emph{universal}, if every countable model embeds elementarily into $\mathcal{M}$.
\end{definition}

\begin{theorem}[Ryll-Nardzewski]
    Let $T$ complete and $L$-countable. Then, $T$ is $\aleph_0$-categorical $\iff$ every countable model is prime $\iff$ every countable model is atomic $\iff$ every type is principal $\iff$ there are only finitely many $n$-types $\iff$ $n$-formulas $\phi(\vec{x})$ up to $T$ equivalence is finite $\iff$ \\every countable model is universal $\iff$ a countable model is prime and universal $\iff$ \\ every countable model is universal and homogeneous. Proof: $(4) \implies (5)$: pick $\phi_p$, $T \models \forall \lor \phi_p$. By compactness and negations, finite. $(5) \implies (6)$, $\phi \mapsto^\Phi \{p \mid \phi \in p\}$, show $\Phi(\phi_1) = \Phi(\phi_2)$ iff $\phi_1 \leftrightarrow \phi_2$. $(6) \implies (4)$ easy. Then, $(1) \implies (9)$, $(7) \implies (9)$, $(7) \implies (4)$ and $(8) \implies (2)$. $\mathcal{M}$ prime \& universal: $\mathcal{N} \xrightarrow{g} \mathcal{M} \xrightarrow{h} \mathcal{N}'$ elementarily, so $\mathcal{N}$ is prime.
\end{theorem}

\begin{definition}
    A \emph{saturated} model is a model that realises all $n$-types and is homogeneous. Equivalently: \\
    If $\mathcal{M}$ is saturated, for all $B \subseteq \mathcal{M}$ and $\abs{B} < \abs{\mathcal{M}}$, $\mathcal{M}_B$ realises all 1-types of $\mathrm{Th}(\mathcal{M}_B)$.
\end{definition}

\begin{proposition}
    If $\mathcal{M}$ is saturated and countable, it is universal and unique up to isomorphism.
\end{proposition}

\begin{definition}
    A group $G$ applied to a $G$-set is oligomorphic if there are finitely many orbits of $G$.
\end{definition}
\begin{proposition}
    $T$ is $\aleph_0$-categorical if and only if for a countable $\mathcal{M}$, $\mathrm{Aut}(\mathcal{M})$ is oligomorphic.
\end{proposition}

\end{document}

